\documentclass[final,xcolor=dvipsnames]{beamer}
%
% Choose how your presentation looks.
%
% For more themes, color themes and font themes, see: 
% http://deic.uab.es/~iblanes/beamer_gallery/index_by_theme.html
%
\mode<presentation>
{
  \usetheme{Boadilla}      % or try Darmstadt, Madrid, Warsaw, ...
  \usecolortheme{default} % or try albatross, beaver, crane, ...
  \usefonttheme{structurebold}  % or try serif, structurebold, ...
  \setbeamertemplate{navigation symbols}{}
  \setbeamertemplate{caption}[numbered]
} 

\usepackage[english]{babel}
\usepackage[utf8x]{inputenc}
\usepackage[]{caption}
\usepackage{tikz}


%the-sun-is-in-love-with-the-moon-collab-with-blaak.jpg

%\titlegraphic{\includegraphics[]{the-sun-is-in-love-with-the-moon-collab-with-blaak.jpg} \hspace{2cm} \includegraphics[width=2cm]}}
\title[Introduction to Environmental Modelling]{Introduction to Environmental Modelling}
\subtitle{Face-It Summer School 2019 : Marine Biogeochemical Cycling: from measurements to modelling}
\author[A. Capet]{A. Capet, M.Grégoire, K. Soetaert - acapet@ulg.ac.be} %Logo de MAST +logo interface sur chaque slide et page de garde interface?
\institute[http://labos.ulg.ac.be/mast/]{MAST}
\date[Oct 2019]

\AtBeginSection[]{
  \begin{frame}
  \vfill
  \centering
  \begin{beamercolorbox}[sep=8pt,center,shadow=true,rounded=true]{title}
    \usebeamerfont{title}\insertsectionhead\par%
  \end{beamercolorbox}
  \vfill
  \end{frame}
}

\begin{document}

% \usebackgroundtemplate{%             declare it
% \tikz[overlay,remember picture] \node[opacity=0.6, at=(current page.center)] {
%   \includegraphics[height=\paperheight,width=\paperwidth]{the-sun-is-in-love-with-the-moon-collab-with-blaak.jpg}
%   };
%   % {Great_Barrier_Reef_Australia}};
% }

\begin{frame}
  \titlepage
\end{frame}

% \usebackgroundtemplate{ }  

\section{Basics Concepts}
\begin{frame}{What is a model ?}
\visible<2>{
 A \alert{simplified} representation of a complex phenomenon.} 
\end{frame}

\begin{frame}{How simple ?}
\centering{Arguments in favor of:}
\begin{columns}
\begin{column}{.5\framewidth}
\centering{\alert{Simplicity}}
\end{column}
\begin{column}{.5\framewidth}
\centering{\alert{Complexity}}
\end{column}
\end{columns}
\end{frame}


\section{Elements of a model}
\subsection{Research Questions}

\begin{frame}
\begin{block}{Aim of sutdy}
\begin{itemize}
    \item to develop a one-dimensional model that simulates the
\end{itemize} 
dynamics of different local processes occurring in the water column of a pelagic
system, under the presence of an installed turbine that is covered with the blue
mussel Mytilus edulis.

Our study mainly focuses on
the increased production of organic matter (faeces and pseudofaeces),
food depletion by the growth of biofouling organisms
impacts on biogeochemical processes via respiration and excretion. 

The main objectives are: 

i) to test how deep the blue mussels can grow in the structure under mixed and stratified conditions of the \textbf<2>{water} column,
ii) to determine if there will be local depletion of food resources
such as phytoplankton, zooplankton and detritus,
iii) to test if the mussels that grow on the seabed will have the same effect as the mussel beds located on the upper layers of the structure iv) to test if the type of turbine and the distance between the turbines has an effect on the accumulation of mussel’s biomass and on ecosystem and biogeochemical dynamics.
\end{block}
\end{frame}

\subsection{Scales}


\begin{frame}
\begin{block}{Aim of sutdy}
\begin{itemize}
    \item to develop a one-dimensional model that simulates the
\end{itemize} 
dynamics of different local processes occurring in the water column of a pelagic
system, under the presence of an installed turbine that is covered with the blue
mussel Mytilus edulis.

Our study mainly focuses on
the increased production of organic matter (faeces and pseudofaeces),
food depletion by the growth of biofouling organisms
impacts on biogeochemical processes via respiration and excretion. 

The main objectives are: 

i) to test how deep the blue mussels can grow in the structure under mixed and stratified conditions of the \textbf<2>{water} column,
ii) to determine if there will be local depletion of food resources
such as phytoplankton, zooplankton and detritus,
iii) to test if the mussels that grow on the seabed will have the same effect as the mussel beds located on the upper layers of the structure iv) to test if the type of turbine and the distance between the turbines has an effect on the accumulation of mussel’s biomass and on ecosystem and biogeochemical dynamics.
\end{block}
\end{frame}



\subsection{State Variables}

\begin{frame}
 Tsting GIT
\end{frame}

\subsection{Dynamic Equation}
\subsection{Conservation ?}
\subsection{External Inputs}
\subsection{'Observable Variables'}
\subsection{Integration}



\section{A practical example : Mussels on Wind Farms}
\subsection{State Variables}

\subsection{Fluxes}

\subsection{Integration}


\begin{frame}{What are ressources ?}
 A factor that
 \begin{itemize}
 \item has a given availability
 \item leads to higher growth as availability increases
 \item is consumed by the population(s)
 \end{itemize}
 Multiple factors can be considered as resources if they meet the above criteria for some range of other factors availability.
\end{frame}

\end{document}

