\documentclass[final,xcolor=dvipsnames]{beamer}
%
% Choose how your presentation looks.
%
% For more themes, color themes and font themes, see: 
% http://deic.uab.es/~iblanes/beamer_gallery/index_by_theme.html
%
\mode<presentation>
{
  \usetheme{Boadilla}      % or try Darmstadt, Madrid, Warsaw, ...
  \usecolortheme{default} % or try albatross, beaver, crane, ...
  \usefonttheme{structurebold}  % or try serif, structurebold, ...
  \setbeamertemplate{navigation symbols}{}
  \setbeamertemplate{caption}[numbered]
} 

\usepackage[english]{babel}
\usepackage[utf8x]{inputenc}
\usepackage[]{caption}
\usepackage{tikz}


%the-sun-is-in-love-with-the-moon-collab-with-blaak.jpg

%\titlegraphic{\includegraphics[]{the-sun-is-in-love-with-the-moon-collab-with-blaak.jpg} \hspace{2cm} \includegraphics[width=2cm]}}
\title[Introduction to Environmental Modelling]{Introduction to Environmental Modelling}
\subtitle{Face-It Summer School 2019 : Marine Biogeochemical Cycling: from measurements to modelling}
\author[A. Capet]{A. Capet, M.Grégoire, K. Soetaert - acapet@ulg.ac.be} %Logo de MAST +logo interface sur chaque slide et page de garde interface?
\institute[http://labos.ulg.ac.be/mast/]{MAST}
\date[Oct 2019]

\AtBeginSection[]{
  \begin{frame}
  \vfill
  \centering
  \begin{beamercolorbox}[sep=8pt,center,shadow=true,rounded=true]{title}
    \usebeamerfont{title}\insertsectionhead\par%
  \end{beamercolorbox}
  \vfill
  \end{frame}
}

\begin{document}

% \usebackgroundtemplate{%             declare it
% \tikz[overlay,remember picture] \node[opacity=0.6, at=(current page.center)] {
%   \includegraphics[height=\paperheight,width=\paperwidth]{the-sun-is-in-love-with-the-moon-collab-with-blaak.jpg}
%   };
%   % {Great_Barrier_Reef_Australia}};
% }

\begin{frame}
  \titlepage
\end{frame}

% \usebackgroundtemplate{ }  

\section{Basics Concepts}
\begin{frame}{What is a model ?}
\visible<2>{
 A \alert{simplified} representation of a complex phenomenon.} 
\end{frame}

\begin{frame}{How simple ?}
\centering{Arguments in favor of:}
\begin{columns}
\begin{column}{.5\framewidth}
\centering{\alert{Simplicity}}
\end{column}
\begin{column}{.5\framewidth}
\centering{\alert{Complexity}}
\end{column}
\end{columns}
\end{frame}


\section{Elements of a model}
\subsection{Research Questions}
\subsection{Scales}
\subsection{State Variables}
\subsection{Dynamic Equation}
\subsection{Conservation ?}
\subsection{External Inputs}
\subsection{'Observable Variables'}
\subsection{Integration}



\section{A practical example : Mussels on Wind Farms}
\subsection{State Variables}

\subsection{Fluxes}

\subsection{Integration}


\begin{frame}{What are ressources ?}
 A factor that
 \begin{itemize}
 \item has a given availability
 \item leads to higher growth as availability increases
 \item is consumed by the population(s)
 \end{itemize}
 Multiple factors can be considered as resources if they meet the above criteria for some range of other factors availability.
\end{frame}

\end{document}

